\documentclass{jarticle}
\usepackage{listings}
\title{\\ \\ FunCobal \\ Basical Specification Document \\ Ver. 1.1.0}
\author{Takuya Matsunaga \\ NAIPSS-CMSS: IntelligenceResearch}
\date{24/4/2020}
\pagestyle{plain}
\pagenumbering{Roman}

\begin{document}
\maketitle
\tableofcontents
\newpage
\section{Introduction}
\subsection{Introduction}
\subsection{Background}
\section{Outline}
\par
 Style: Object Orenteted Language with Some Functional Style.
\subsection{Name}
The name "FunCobal" is named by Mr. Suwawa. According to his saying, the origin is "Fun" from "with Some Functional Style" and hope that the language will be fantastic, and "Cobal" from "Cobaltia" , the alias name of Haruka Sato(Takuya Matsunaga), chif developer of the language.
\subsection{Runtime and Development Platform}
\subsubsection{Outline}
\subsubsection{Local Runtime Platform}
\subsubsection{Serveral Runtime Platform}
\subsubsection{Local Development Platform}
\subsubsection{Serveral Development Platform}
\subsubsection{FunCobal Library Platform}
\newpage
\subsection{Class and Object}
\subsection{Variable and Constant}
Each Variables and Constants consist of a Modifier and a Type.
\par
Modifiers of Variable and Constant are bellow.
\begin{itemize}
  \item item1
  \item item2
  \item ...
  \item itemN
\end{itemize}

\begin{itemize}
  \item item1
  \item item2
  \item ...
  \item itemN
\end{itemize}
\subsection{Operator and Equalation}
\subsection{Type Casting on Equation}
\subsection{Array}
\subsubsection{Array}
\subsubsection{Array}
\subsection{Comments}
\newpage
\section{Classes and Methods}
\subsection{Structure and Style}
\subsection{Modifiers}
 The language have some Modifiers for Classes bellow.
 \begin{itemize}
  \item abstruct ...
  \item final ...
  \item public ...
\end{itemize}
 and also for Methods bellow.
\begin{itemize}
 \item abstruct
 \item final
 \item public
\end{itemize}
 and also for Variables, and  Constants, bellow.
\begin{itemize}
 \item private ...
 \item pritected ...
 \item public ...
 \item static ...
 \item transient ...
 \item volatile ...
\end{itemize}
\end{document}
